  \subsection{Standard Reference Frame (StdRefFrame)}
  \label{sect:StdRefFrame}

  \textbf{BaseURL}:  http://www.ivoa.net/rdf \newline
  \textbf{Vocabulary}: refframe \newline

  We include here the initial Standard Reference Frame Vocabulary.  The formal list is stored and maintained as a controlled vocabulary external to this document at the URL listed above.  The URI for each term is built as <BaseURL>/<Vocabulary>/<Term>, where Term is one of:

  \paragraph{EQUATORIAL Frames}
  \small
  \begin{itemize}
    \item[\textbf{ICRS}]:  International Celestial Reference System
    \item[\textbf{FK4}]:  Fundamental Katalog, system 4; Besselian. \newline Requires Equinox; default B1950.0
    \item[\textbf{FK5}]:  Fundamental Katalog, system 5; Julian. \newline Requires Equinox; default J2000.0
  \end{itemize}
  \normalsize

  \paragraph{ECLIPTIC Frames}
  \small
  \begin{itemize}
    \item[\textbf{ECLIPTIC}]:  Ecliptic coordinates
  \end{itemize}
  \normalsize

  \paragraph{GALACTIC Frames}
  \small
  \begin{itemize}
    \item[\textbf{GALACTIC\_I}]:  Old Galactic coordinates
    \item[\textbf{GALACTIC}]: "New" Galactic coordinates
    \item[\textbf{SUPER\_GALACTIC}]: Super-galactic coordinates \newline pole at GALACTIC (47.37, +6.32); origin at GALACTIC (137.37, 0)
  \end{itemize}
  \normalsize

  \paragraph{OTHER Frames}
  \small
  \begin{itemize}
    \item[\textbf{AZ\_EL}]:  Local azimuth and elevation \newline Ground-based observations; Azimuth from North through East
    \item[\textbf{BODY}]:  Generic "BODY" coordinates
    \item[\textbf{UNKNOWN}]:  Unknown reference frame \newline Only to be used as a last resort or for simulations. The client is responsible for assigning a suitable default.
  \end{itemize}
  \normalsize
